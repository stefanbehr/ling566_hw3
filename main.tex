\documentclass{article}
\usepackage[utf8]{inputenc}
\usepackage[left=1in,right=1in,top=1in,bottom=1in]{geometry}

\title{LING 566: Homework 3}
\author{Stefan Behr}
\date{\today}

\usepackage{qtree}
\usepackage{parskip}
\usepackage{avm}
\usepackage{graphicx}
\avmfont{\sc}
\avmvalfont{\rm}
\avmsortfont{\it}

\parindent=0in
\pagestyle{empty}

\begin{document}

\maketitle

\renewcommand{\thesubsection}{\Alph{subsection}.}

\section{Chapter 6, Problem 2: Spanish NPs II}

\subsection{Using the MOD feature to specify which nouns the adjective can modify, give a lexical entry for `peque\~nos'. Be sure to specify both SYN and SEM features.}

\newcommand{\pequenos}{
                    \[
                        \it{word} \\
                        SYN &   \[
                                    \it{syn-cat} \\
                                    HEAD &  \[
                                                \it{adj} \\
                                                AGR & \@1   \[
                                                                \it{agr-cat} \\
                                                                PER & 3rd \\
                                                                NUM & pl \\
                                                                GEND & masc \\
                                                            \]\\
                                            \]\\
                                    VAL &   \[
                                                \it{val-cat} \\
                                                SPR & \< \> \\
                                                COMPS & \< \> \\
                                                MOD & \<    \avml \hfil N \hfil \\
                                                                \[
                                                                    AGR & \@1 \\
                                                                    INDEX & \it{i} \\
                                                                \] \avmr \> \\
                                            \]\\
                                \]\\
                        SEM &   \[
                                    \it{sem-cat} \\
                                    MODE & none \\
                                    INDEX & \it{t} \\
                                    RESTR & \<  \[
                                                    RELN & \bf{peque\~nos} \\
                                                    ARG & \it{i} \\
                                                \] \>\\
                                \]\\
                    \]
}

\begin{center}
\begin{avm}
\< \textit{peque\~nos}, \pequenos \>
\end{avm}
\end{center}

\subsection{Assuming the rules we have developed for English are appropriate for Spanish as well, draw a tree for the NP `los ping\"uinos peque\~nos' in (iv). Show values for all features, using tags to show identities required by the grammar.}

\begin{center}[ tree attached on separate page ]\end{center}

\subsection{Explain how the INDEX value of `ping\"uinos' is identified with the argument of the predication introduced by `peque\~nos'. (Your explanation should indicate the role of lexical entries, rules, and principles in enforcing this identity.)}

\par{The feature structure in the lexical entry for \textit{peque\~nos} specifies that the value of its {\sc RESTR} list's sole predication's {\sc ARG} feature, \textit{i}, must be identical to the value of the {\sc INDEX} feature of the sole element in the feature structure's {\sc MOD} list. This ensures that, in order for \textit{peque\~nos} to combine into a phrase with a head that it will modify, by the {\sc Head-Modifier Rule}, that head must identify its {\sc INDEX} value with the {\sc ARG} value of \textit{peque\~nos}. This, consequently, is how \textit{ping\"uinos} gets its {\sc INDEX} value of \textit{i} (in this particular example).}

\section{Chapter 6, Problem 3: English Possessives I}

\subsection{What is the generalization about where the 's of possession appears in English?}

\par{In general in English, within a subject or object NP, \textit{'s} seems to appear directly to the left of the rightmost N in the NP, and directly to the right of the largest possible NP that can be formed, within the top-level object or subject NP, to the left of and excluding the rightmost N. So, for example, in \textit{(ii) Jesse met the president of the university's cousin.}, the object NP \textit{the president of the university's cousin} has an N, \textit{cousin}, directly to the right of \textit{'s}. The largest possible NP that can be formed within the object NP, which excludes \textit{cousin}, is \textit{the president of the university}. So, \textit{'s} attaches to it.}\\

\par{In the ungrammatical example \textit{(iii) *Jesse met the president's of the university cousin.}, the \textit{'s} is attaching to an NP, \textit{the president}, which is not the largest NP that can be formed within the object NP of \textit{met} while excluding the rightmost noun \textit{cousin} --- the PP \textit{of the university} could be included in the NP to which \textit{'s} is attaching in order to make it an NP of the desired maximal size. Furthermore, there is no N directly to the right of the \textit{'s}.}

\subsection{Which of these sentences does it predict should be grammatical, and why?}

With the addition of the new rule, the following sentences are grammatical: \textit{(i)-(v)} and \textit{(vii)}. The following are not: \textit{(vi)}. Explanations for each sentence:\\

\parindent=0in

\textbf{(i)} \textit{Leslie} is an NP which is given [{\sc CASE} poss] by \textit{'s}. This allows \textit{Leslie's} to be used as a determiner for \textit{coffee}, combining into another NP which will be taken by \textit{spilled} as its specifier.\\

\textbf{(ii)} \textit{'s} marks the noun \textit{university} to give it [{\sc CASE} poss]. The determiner \textit{the} can combine with \textit{university's} to create an NP with [{\sc CASE} poss] (which is passed up via the HFP). Then, by our new rule, that NP can be made into a determiner for \textit{cousin}, to form a new NP which will be the object of the preposition \textit{of}. From there, the resulting PP \textit{of the university's cousin}, though nonsensical, can modify \textit{president}, the result of which can take \textit{the} as a specifier to form the object NP of the verb \textit{met}. Finally, the resulting VP takes the NP \textit{Jesse} as its specifier. The resulting sentence is nonsensical, but licensed by the grammar and rule addition.\\

\textbf{(iii)} Here, \textit{'s} marks \textit{president}, making it an N with [{\sc CASE} poss]. The PP \textit{of the university} can modify \textit{president's}. The resulting NOM (which still has [{\sc CASE} poss] via the HFP) can take the leftmost \textit{the} as its specifier, creating an NP with [{\sc CASE} poss], again via the HFP. By our new rule, this NP can be turned into a determiner for specifying the noun \textit{cousin}. The NP that results from that combination is the object of \textit{met}, and \textit{Jesse} is the specifier of the VP \textit{met the president's of the university cousin}. As in \textit{(ii)}, this sentence is nonsensical, but licensed.\\

\textbf{(iv)} \textit{trail's} has [{\sc CASE} poss], it combines with \textit{the} to create an NP with [{\sc CASE} poss], which can be turned into a determiner for \textit{leaves}. The resulting NP can form a PP with \textit{by}, which can in turn modify \textit{growing}. \textit{growing by the trail's leaves} can modify \textit{plant}, and the resulting NOM takes \textit{that} as its specifier to create the verb's object NP, and so on. The resulting sentence actually could be construed to make sense. Either way, it's licensed by the grammar.\\

\textbf{(v)} Here, \textit{growing by the trail} is an AP that can modify the [{\sc CASE} poss] N \textit{plant's} to form a [{\sc CASE} poss] NOM. The NOM takes \textit{that} as its specifier to create an NP with [{\sc CASE} poss] (via the HFP from N to NOM, and from NOM to NP). This NP can turn into a determiner via our new rule, and be used as the specifier for \textit{leaves}, creating the verb's object NP, and so on.\\

\textbf{(vi)} \textit{'s} attempts to attach to the word \textit{to}, whose {\sc HEAD} feature has type \textit{prep}. The feature {\sc CASE} is not appropriate for the type \textit{prep}, according to the type hierarchy in the Chapter 5/6 grammar. Therefore, the attachment of \textit{'s} to \textit{to} is ungrammatical, rendering the sentence in \textit{(vi)} ungrammatical.\\

\textbf{(vii)} In this sentence, \textit{'s} attaches to the noun \textit{person}. \textit{person's} functions as the head of the NP \textit{The person's you were talking to}, which means that [{\sc CASE} poss] is passed from \textit{person's} up to the top level of the aforementioned phrase, allowing the entire phrase to be used as a determiner for \textit{pants}. Combining the two creates the subject NP of the sentence. Therefore, this example is judged to be grammatical, though, unsurprisingly, nonsensical.

\section{Chapter 6, Problem 4}

\newcommand{\apos}{
                    \[
                        \it{word} \\
                        SYN &   \[
                                    \it{syn-cat} \\
                                    HEAD &  \[
                                                \it{det} \\
                                                AGR & \textit{3sing}\\
                                            \]\\
                                    VAL &   \[
                                                \it{val-cat} \\
                                                SPR & \< NP \> \\
                                                COMPS & \< \> \\
                                                MOD & \< \> \\
                                            \]\\
                                \]\\
                        SEM &   \[
                                    \it{sem-cat} \\
                                    MODE & none \\
                                    INDEX & \it{i} \\
                                    RESTR & \<  \[
                                                    RELN & \bf{'s} \\
                                                    BV & \it{i} \\
                                                \] \>\\
                                \]\\
                    \]
}

\subsection{Ignoring semantics for the moment, give the lexical entry for \textit{'s} assuming its analysis as a determiner, and draw a tree for the NP \textit{Kim's brother}.}

\begin{center}
\begin{avm}
\< \textit{'s}, \apos \>
\end{avm}
\end{center}

(The SHAC will take care of the identification of AGR values between \textit{'s} and its specifier.)

\end{document}
